\documentclass{article}
\usepackage{amsmath}
\usepackage{amssymb}
\usepackage{mathtools}
\usepackage{graphicx}
\usepackage{graphicx}
\graphicspath{{/home/rp/Documents/Christian_School/Plots/}}

\DeclarePairedDelimiter{\norm}{\lVert}{\rVert}

\begin{document}

\title{Final Project AE 596\\\large IGM Guidance algorithm}

\author{Christian Engelbrecht}


\maketitle

\newpage
\section{Equations of motion; K1, K2}

For a given set of final states of an ascent vehicle leaving the lunar surface, we can write the equations of motion simply as:\newline
\begin{align*}
\dot{x} &= V_x \\ 
\dot{y} &= V_y \\ 
\dot{V_x} &= a\cos{\beta} \\ 
\dot{V_y} &= -g + a\sin{\beta} \\ 
\end{align*}

Where $\beta$ is the flight angle of the vehicle, and

\begin{equation*} %putting the * in there makes it not be numberes 
a = \frac{T}{m(t)} = \frac{T}{m_0 - \dot{m}(t - t_0)}
\end{equation*}

is the thrust acceleration of the vehicle,for a given thrust $T$.The vehicle ascends from some time 0 to a final time $t_f$, where the final condition of the vehicle is 

\begin{align*}
V_x(t_f) &= V_x^* \\
V_y(t_f) &= V_y^* \\
y(t_f) &= y_f^*
\end{align*}

Using optimal control theory, we now aim to maximize the Hamiltonian $H$ of the system. The Hamiltonian is 

\begin{equation*}
H = P_x V_x + P_yV_y + P_{V_{x}} a\cos(\beta) + P_{V_{y}}(-g +a\sin(\beta))
\end{equation*}

Where $\boldsymbol{P_r}$ is $\binom{P_x}{P_y}$ , and $\boldsymbol{P_V}$ is $\binom{P_{V_{x}}}{P_{V_{y}}}$ are the costate vectors for 
position and velocity, respectively. \newline
We also define a performance index $J$ that will minimize the total flight time 

\begin{equation*}
J = t_f
\end{equation*}

The derivatives for the costate equations are 

\begin{align*}
\dot{P_x}-\frac{\partial H}{\partial x} &= 0 & \dot{P_y}-\frac{\partial H}{\partial y} &= 0 \\
\Rightarrow P_x &= c_1 & \Rightarrow P_y &= c_2\\
\dot{P_{V_{x}}} = -\frac{\partial H}{\partial V_{x}} &= -P_{x}  & \dot{P_{V_{y}}}-\frac{\partial H}{\partial V_{y}} &= -P_{y} \\
\Rightarrow P_{V_{x}} &= c_1t +c3 & \Rightarrow P_{V_{y}} &= c_2t +c4
\end{align*}

Our control variable here is $\beta$, and since this is not a constrained control ($\beta$ can take on any value), the optimility condtition 
$\frac{\partial H}{\partial \beta} = 0$ gives us an expression to relate $\beta$ to the constants found above: 

\begin{align*}
\frac{\partial H}{\partial \beta} =  (\frac{T}{M}) (-P_{V_{x}} \sin\beta + P_{V_{y}}\cos\beta) &= 0 \\
\Rightarrow -P_{V_{x}} \sin\beta + P_{V_{y}}\cos\beta &= 0 \\
P_{V_{x}} tan\beta - P_{V_{y}} &= 0 \\
\tan\beta &= \frac{P_{V_{y}}}{P_{V_{x}}} \\ 
\tan\beta &= \frac{c_2 t + c_4}{c_1 t + c_3}																
\end{align*}

This is the familiar bilinear tangent law. We can now use the transversaility conditions to trim down the number of unknowns. The transversality
condition for $P_{V_{x}}$ is	

\begin{align*}
P_x (t_f) = -\frac{\partial \Phi }{\partial x(t_f)} = -\frac{\partial t_f }{x(t_f)} &= 0 \\
\Rightarrow P_x = c_1 &= 0 
\end{align*}

Similarly, the transversality condition for $P_y$, in a simplified situation where $y(t_f)$ is not specified:

\begin{align*}
P_y (t_f) = -\frac{\partial \Phi }{\partial y(t_f)} = -\frac{\partial t_f }{y(t_f)} &= 0 \\
\Rightarrow P_y = c_2 &= 0 \\
\Aboxed{\therefore\tan\beta &= \frac{c_4}{c_3} = constant \equiv \tan\tilde{\chi}}
\end{align*}

We can now integrate the equations of motion to find explicit closed form solutions for $V_x, V_y, x,y$. Integrating $V_x$ and $V_y$, recalling $\dot{m} = -\frac{T}{g_0 I_{sp}}$: 

\begin{align*}
V_x(t) &= \cos{\beta}\int_{t_0}^{t} \frac{Td\tau}{m-\dot{m}(t_0 - \tau)}\notag \\
&= T\cos{\beta} \cdot \frac{\ln (m_0 + \dot{m}(t_0 - \tau)}{\dot{m}}\Big|_{\tau = t_0}^{\tau = t}\notag \\
&= -g_0I_{sp}\cos\beta \cdot [\ln(m_0 + \dot{m}(t_0 + t)) - \ln(m_0)] + V_x(t_0)\notag \\
&= g_0I_{sp}\cos\beta\ln\left(\frac{m_0}{m_0 + \dot{m}(t_0 + t)}\right) + V_x(t_0)
\end{align*}

\begin{align*}
V_y(t) &= \sin{\beta}\int_{t_0}^{t} \frac{Td\tau}{m-\dot{m}(t_0 - \tau)} + \int_{t_0}^{t} gd\tau\notag\\
&= T\sin{\beta} \left (\frac{\ln (m_0 + \dot{m}(t_0 - \tau)}{\dot{m}}\Big|_{\tau = t_0}^{\tau = t}  + g\tau\Big|_{\tau = t_0}^{\tau = t} \right )\notag \\
&= -g_0I_{sp}\sin\beta  [\ln(m_0 + \dot{m}(t_0 + t)) - \ln(m_0) + g(t - t_0)] + V_y(t_0)\notag \\
&= g_0I_{sp}\sin\beta\ln\left(\frac{m_0}{m_0 + \dot{m}(t_0 + t)}\right)  - g(t - t_0)+ V_y(t_0)
\end{align*}

We now use the end conditions $V_x(t_f) = V_x^*$ and $V_y(t_f) = V_y^*$. We define the temporary variable $\gamma \equiv g_0I_{sp}\ln\left(\frac{m_0}{m_0 + \dot{m}(t_0 + t)}\right)$

\begin{align}
\Rightarrow V_y^* + g(t_f - t_0) - V_y(t_0) &= \gamma\sin\tilde\chi \\
V_x^* - V_x(t_0) & = \gamma\cos\tilde\chi
\end{align}

Then divide $(1)$ by $(2)$: 

\begin{equation}
\Rightarrow \tan\tilde\chi = \frac{V_y^* - V_y(t_0) + g(t_f - t_0)}{V_x^* - V_x(t_0)}
\label{eq:three}
\end{equation}

This equation now only depends on $t_f - t_0$, which is called the Time to Go: $t_{go}$. We will use this equation to determine the value for $\tilde\chi$ given a value for 
$t_{go}$ \newline
We now consider again the linear tangent law, with the full constraint on final vertical position (i.e. the transversality condition for $P_y$ does not apply), the bilinear 
tangent law is now the linear tangent law, where the constants $c_2, c_3 $ and $c_4$ are recast to $k1,k_0$. We consider the tangenet law not at a specific time $t$ but at 
the time interval $t - t_0$

\begin{equation*}
\tan\beta = k_1(t - t_0) + k_0
\end{equation*}
We discovered above that $\tan\beta$ is constant when some assumptions are taken into account, and equal to $\tilde\chi$ in those cases. Now, with those assumptions relaxed, 
we say that the linear tangent law still applies but there is a slight offset in $\tilde\chi$: that is, $k_0 = \tilde\chi + k_2$

\begin{equation*}
\tan\beta = k_1(t - t_0) + \tilde\chi + k_2
\end{equation*}

If we further assume that offset is very small, i.e $\left | k_1 (t - t_0) + k_2 \right | << 1$, and $\beta$ is very small, then we can use the small angle approximation

\begin{equation}
\beta = k_1(t - t_0) + \tilde\chi +k_2
\label{eq:four}
\end{equation}

This angle $\beta$ is the angle necessary for integrating the equations of motion. We need in particular it's sine and cosine. Using the double angle identity, we can say
\begin{align*}
\sin(A+B) &= \sin(A)\cos(B) + \cos(A)\sin(B)\\
\cos(A+B) &= \cos(A)\cos(B) - \sin(A)\sin(B)\\
\Rightarrow \sin(\beta) &= \sin(\tilde\chi+ (k_1(t-t_0)+k_2))\\
&=\sin(\tilde\chi)\cos(k_1(t-t_0)+k_2) + \cos(\tilde\chi)\sin(k_1(t-t_0)+k_2)\\
&\approx\sin(\tilde\chi) + k_1(t - t_0)\cos(\tilde\chi) + k_2\cos(\tilde\chi)\\
\cos(\beta) &= \cos(\tilde\chi)\cos(k_1(t-t_0)+k_2) -\sin(\tilde\chi)\sin(k_1(t-t_0)+k_2)\\
&\approx cos(\tilde\chi) - k_2\sin(\tilde\chi) - k_1(t-t_0)\sin(\tilde\chi)
\end{align*}

At this point the EOMs are entirely integrable, given a $t_{go}$ To find this quantity, we use an iterative approach using the one dimensional rocket equation and the
incremental velocity at a given time $t_0$. First, a time is chosen at random and a $\Delta V$ is calculated given the state of the velocity at that time (equation 5). Then
the velocity is calculated using the Tsiolkovski rocket equation (equations 6 and 7): if the velocities match, the initial time chosen is a solution to the set of equations. If not,
the time is incremented and the conditions is checked again. 

\begin{align}
\Delta V &= \sqrt{(V_x^* - V_x)^2 +(V_y^* - V_y +g(t_{go)}))}\\
\Delta V &=g_0 I_{sp}\ln{\frac{m_0}{m}} ,\ m = m_0 - \dot{m}t_{go} \\
\Rightarrow t_{go} &= \frac{-m_0}{\dot{m}} (e^{\frac{-\Delta V}{I_{sp}*g_0}} - 1) 
\end{align}
 
Now with all tools in hand, the equations of motion can properly be solved for each time step, and an iterative guidance program can be made. Starting by integrating $V_y$

\begin{align*}
\dot{V_y}&= -g + a \sin\beta\\
&= -g + \frac{T}{{m_0 - {\dot{m}}(t-t_0)}} *(\sin\tilde\chi + k_2\cos\tilde\chi + k_1\cos\tilde\chi(t - t_0)) \\
\therefore V_y(t) &= -g(t-t_0) + T\int_{t_0}^t\frac{\sin\tilde\chi+k_2\cos\tilde\chi}{m_0 - {\dot{m}}(\tau-t_0)} d\tau + T\int_{t_0}^t\frac{cos\tilde\chi k_1 (\tau - t_0)}{{m_0 - {\dot{m}}(\tau-t_0)}}d\tau +V_y(t_0)
\end{align*}

Defining the constants $a_{22},a_{21}$ and $a_{20}$ as the coeffiecients of $k_2$,$k_1$ and the constant coefficient, respectively:

\begin{align*}
a_{22} &= T\cos{\tilde\chi}\int_{t_0}^t \frac{d\tau}{m_0 - \dot{m}(\tau - t_0)} \\ 
&= \frac{T\cos\tilde\chi}{\dot{m}} \ln\frac{m_0}{m_0 - \dot{m}(t-t_0)} \\
a_{21} &= T\cos{\tilde\chi}\int_{t_0}^t \frac{(\tau - t_0)d\tau}{m_0 - \dot{m}(\tau - t_0)} \\ 
&= T\cos{\tilde\chi} (\frac{m_0\ln({\frac{m_0}{m_0 - \dot{m}(t_0 - t)}})}{\dot{m}^2} - \frac{(t + t_0)}{\dot{m}}) \\
a_{20} &= V_y(t_0) - \int_{t_0}^t \frac{d\tau}{m_0 - \dot{m}(\tau - t_0)} - g\\
 &= \frac{T\sin{\tilde\chi}}{\dot{m}} \ln{\frac{m_0}{m_0 - \dot{m}(t_0 - t)}} - g(t - t_0)
\end{align*}

\begin{equation*}
\Rightarrow V_y(t) = a_{22}k_2 + a_{21}k_1 + a_{20}
\end{equation*}

For the longer derivations of these and the following coefficients, please see the attached paper to this report. 
Now we can integrate this equation for $V_y(t)$ one more time to get $y(t)$

\begin{align*}
y(t) &= y(t_0) + \int_{t_0}^tV_y(\tau)d\tau \\ 
&= y(t_0) + \int_{t_0}^ta_{22}k_2 dt + \int_{t_0}^ta_{21}k_1 dt + \int_{t_0}^t +a_{20}dt \\
&= y(t_0) +a_{12}k_2 + a_{11}k_1 + a_{10}
\end{align*}

Definining the coefficients $a_{11}$, $a_{12}$ , $a_{10}$

\begin{align*}
a_{10} &= \int_{t_0}^t a_{20}dt \\ 
&= V_y(t_0)(t-t_0) - \frac{g}{2}(t-t_0)^2 + \frac{T\sin\tilde\chi}{|\dot{m}|}\Big[\frac{(t-t_0)|\dot{m}| - (m_0|\dot{m}|(t-t_0)\ln{\frac{m_0}{m_0 - |\dot{m}|(t-t_0)}}}{y} \Big] \\
a_{11} &= \int_{t_0}^t a_{21}k_1 dt \\ 
&= T\cos\tilde\chi\Big[\frac{m_0}{\dot{m}^2} \Big(\frac{(t-t_0)|\dot{m}| - (m_0 - |\dot{m}|(t-t_0)ln\frac{m_0}{m_0 - |\dot{m}|(t-t_0)} )}{|\dot{m}|}\Big) - \frac{(t-t_0)^2}{2|\dot{m}|}\Big]\\
a_{12} &= \int_{t_0}^t a_{22}k_1 dt\\
&= T\cos{\tilde\chi}\Big[|\dot{m}|(t-t_0) - (m_0 - |\dot{m}|(t-t_0)\ln{\frac{m_0}{m_0 - |\dot{m}|(t-t_0)}}\Big]
\end{align*}

These derivations are also attached to this report, and simplified versions are implemented in the code \newline
The coefficients of $k_1$ $k_2$ are now explicit functions of time, and the current state of the vehicle. We can use this, along with the explicit solution for $V_y$ found above, 
to set up a system of equations to solve for the constants $k_1$ and $k_2$, given a $t_{go}$ from the iterative method also found above. \newline
That is, replacing $\beta$ with $\tilde\chi$ in the explicit equation for $V_y$:

\begin{align*}
V_y(t_f) &= \left (a_{22}  \right )_{t_f} k_2 +\left (a_{21}  \right )_{t_f} k_1 + \left (a_{20}  \right )_{t_f} \\
&= g_0I_{sp}\sin\tilde\chi\ln\left(\frac{m_0}{m_0 - |\dot{m}|(t_{go})}\right)  - gt_{go}+ V_y^*
\end{align*}

We also use the terminal constraint $y^*$

\begin{align*}
y(t_f) &= y_f^* \\
& = \left (a_{12}  \right )_{t_f} k_2 +\left (a_{11}  \right )_{t_f} k_1 + \left (a_{10}  \right )_{t_f} \\
\end{align*}
\begin{equation*}
\therefore
\begin{bmatrix} a_{11} & a_{12} \\ a_{21} & a_{22} \end{bmatrix}
\begin{bmatrix} k_1 \\k_2 \end{bmatrix}
=\begin{bmatrix} b_1 \\b_2 \end{bmatrix}
\end{equation*}
For $b_1$,$b_2$: 	

\begin{align*}
b_1 &= g_0I_{sp}\sin\tilde\chi\ln\left(\frac{m_0}{m_0 - |\dot{m}|(t_{go})}\right)  - gt_{go}+ V_y^* - a_{20} \\
b_2 &= y_f^{*} +V_y^*  - t_{go} - a_{10} 
\end{align*}
This is the linear system which is solved at every timestep, to find the required flight angle $\beta$ from equation 4 above. 

\section{Mission 1 results}

Unfortunately, my code did not work to implement the above routines. The possible sources of error include the derivation of the constants, as well as 
general coding bugs which were not squashed in time. The following plots and code are included only as a show of effort, and obviously do not reflect the 
actual accuracy of the real world IGM guidance. 
\newline
The conditions for the first mission are: 
\begin{equation*}
x(0) = y(0) = V_x(0) = V_y(0) = 0 ,m(0) = 17513kg
\end{equation*}

\begin{equation*}
V_x^* = 1594.6 \frac{m}{s}, V_y^* = 0 \frac{m}{s} , y_f^* = 185000 m 
\end{equation*}

For both missions, the vehicle's characteristics were 

\begin{equation*}
T = 60051.0 , \dot{m} = -19.118 kg/s
\end{equation*}

\begin{figure} 
\includegraphics[width = \textwidth]{M1_XvY.png}
\includegraphics[width = \textwidth]{M1_BetavT.png}

\end{figure}

\begin{figure}
\includegraphics[width = \textwidth]{M1_Yvt.png}
\includegraphics[width = \textwidth]{M1_Vvt.png}

\end{figure}

\begin{figure} 
\includegraphics[width = \textwidth]{M1_gammavt.png}
\end{figure}

\clearpage
The plots for mission 2 do not look much better, unfortunately. 
The conditions for that mission are
\begin{equation*}
x(0) = y(0) = V_x(0) = V_y(0) = 0 ,m(0) = 17513kg
\end{equation*}

\begin{equation*}
V_x^* = 1700 \frac{m}{s}, V_y^* = 0 \frac{m}{s} , y_f^* = 10000 m 
\end{equation*}

\begin{figure} 
\includegraphics[width = \textwidth]{M2_XvY.png}
\includegraphics[width = \textwidth]{M2_BetavT.png}

\end{figure}

\begin{figure} 
\includegraphics[width = \textwidth]{M2_Yvt.png}
\includegraphics[width = \textwidth]{M2_Vvt.png}
\end{figure}

\begin{figure} 
\includegraphics[width = \textwidth]{M2_gammavt.png}
\end{figure}

\end{document}


