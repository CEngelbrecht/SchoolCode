\documentclass{article}
\usepackage{amsmath}
\usepackage{amssymb}
\usepackage{mathtools}
\usepackage{graphicx}
\begin{document}

\title{Final Project AE 569\\\large IGM Guidance algorithm}

\author{Christian Engelbrecht}


\maketitle

\newpage
\section{Equations of motion; K1, K2}

For a given set of final states of an ascent vehicle leaving the lunar surface, we can write the equations of motion simply as:\newline
\begin{align*}
\dot{x} &= V_x \\ 
\dot{y} &= V_y \\ 
\dot{V_x} &= a\cos{\beta} \\ 
\dot{V_y} &= -g + a\sin{\beta} \\ 
\end{align*}

Where $\beta$ is the flight angle of the vehicle, and

\begin{equation*} %putting the * in there makes it not be numberes 
a = \frac{T}{m(t)} = \frac{T}{m_0 - \dot{m}(t - t_0)}
\end{equation*}

is the thrust acceleration of the vehicle,for a given thrust $T$.The vehicle ascends from some time 0 to a final time $t_f$, where the final condition of the vehicle is 

\begin{align*}
V_x(t_f) &= V_x^* \\
V_y(t_f) &= V_y^* \\
y(t_f) &= y_f^*
\end{align*}

Using optimal control theory, we now aim to maximize the Hamiltonian $H$ of the system. The Hamiltonian is 

\begin{equation*}
H = P_x V_x + P_yV_y + P_{V_{x}} a\cos(\beta) + P_{V_{y}}(-g +a\sin(\beta))
\end{equation*}

Where $\boldsymbol{P_r}$ is $\binom{P_x}{P_y}$ , and $\boldsymbol{P_V}$ is $\binom{P_{V_{x}}}{P_{V_{y}}}$ are the costate vectors for 
position and velocity, respectively. \newline
We also define a performance index $J$ that will minimize the total flight time 

\begin{equation*}
J = t_f
\end{equation*}

The derivatives for the costate equations are 

\begin{align*}
\dot{P_x}-\frac{\partial H}{\partial x} &= 0 & \dot{P_y}-\frac{\partial H}{\partial y} &= 0 \\
\Rightarrow P_x &= c_1 & \Rightarrow P_y &= c_2\\
\dot{P_{V_{x}}} = -\frac{\partial H}{\partial V_{x}} &= -P_{x}  & \dot{P_{V_{y}}}-\frac{\partial H}{\partial V_{y}} &= -P_{y} \\
\Rightarrow P_{V_{x}} &= c_1t +c3 & \Rightarrow P_{V_{y}} &= c_2t +c4
\end{align*}

Our control variable here is $\beta$, and since this is not a constrained control ($\beta$ can take on any value), the optimility condtition 
$\frac{\partial H}{\partial \beta} = 0$ gives us an expression to relate $\beta$ to the constants found above: 

\begin{align*}
\frac{\partial H}{\partial \beta} =  (\frac{T}{M}) (-P_{V_{x}} \sin\beta + P_{V_{y}}\cos\beta) &= 0 \\
\Rightarrow -P_{V_{x}} \sin\beta + P_{V_{y}}\cos\beta &= 0 \\
P_{V_{x}} tan\beta - P_{V_{y}} &= 0 \\
\tan\beta &= \frac{P_{V_{y}}}{P_{V_{x}}} \\ 
\tan\beta &= \frac{c_2 t + c_4}{c_1 t + c_3}																
\end{align*}

This is the familiar bilinear tangent law. We can now use the transversaility conditions to trim down the number of unknowns. The transversality
condition for $P_{V_{x}}$ is	

\begin{align*}
P_x (t_f) = -\frac{\partial \Phi }{\partial x(t_f)} = -\frac{\partial t_f }{x(t_f)} &= 0 \\
\Rightarrow P_x = c_1 &= 0 
\end{align*}

Similarly, the transversality condition for $P_y$, in a simplified situation where $y(t_f)$ is not specified:

\begin{align*}
P_y (t_f) = -\frac{\partial \Phi }{\partial y(t_f)} = -\frac{\partial t_f }{y(t_f)} &= 0 \\
\Rightarrow P_y = c_2 &= 0 \\
\Aboxed{\therefore\tan\beta &= \frac{c_4}{c_3} = constant \equiv \tan\tilde{\chi}}
\end{align*}

We can now integrate the equations of motion to find explicit closed form solutions for $V_x, V_y, x,y$. Integrating $V_x$ and $V_y$, recalling $\dot{m} = -\frac{T}{g_0 I_{sp}}$: 

\begin{align*}
V_x(t) &= \cos{\beta}\int_{t_0}^{t} \frac{Td\tau}{m-\dot{m}(t_0 - \tau)}\notag \\
&= T\cos{\beta} \cdot \frac{\ln (m_0 + \dot{m}(t_0 - \tau)}{\dot{m}}\Big|_{\tau = t_0}^{\tau = t}\notag \\
&= -g_0I_{sp}\cos\beta \cdot [\ln(m_0 + \dot{m}(t_0 + t)) - \ln(m_0)] + V_x(t_0)\notag \\
&= g_0I_{sp}\cos\beta\ln\left(\frac{m_0}{m_0 + \dot{m}(t_0 + t)}\right) + V_x(t_0)
\end{align*}

\begin{align*}
V_y(t) &= \sin{\beta}\int_{t_0}^{t} \frac{Td\tau}{m-\dot{m}(t_0 - \tau)} + \int_{t_0}^{t} gd\tau\notag\\
&= T\sin{\beta} \left (\frac{\ln (m_0 + \dot{m}(t_0 - \tau)}{\dot{m}}\Big|_{\tau = t_0}^{\tau = t}  + g\tau\Big|_{\tau = t_0}^{\tau = t} \right )\notag \\
&= -g_0I_{sp}\sin\beta  [\ln(m_0 + \dot{m}(t_0 + t)) - \ln(m_0) + g(t - t_0)] + V_y(t_0)\notag \\
&= g_0I_{sp}\sin\beta\ln\left(\frac{m_0}{m_0 + \dot{m}(t_0 + t)}\right)  - g(t - t_0)+ V_y(t_0)
\end{align*}

We now use the end conditions $V_x(t_f) = V_x^*$ and $V_y(t_f) = V_y^*$. We define the temporary variable $\gamma \equiv g_0I_{sp}\ln\left(\frac{m_0}{m_0 + \dot{m}(t_0 + t)}\right)$

\begin{align}
\Rightarrow V_y^* + g(t_f - t_0) - V_y(t_0) &= \gamma\sin\tilde\chi \\
V_x^* - V_x(t_0) & = \gamma\cos\tilde\chi
\end{align}

Then divide $(1)$ by $(2)$: 

\begin{equation}
\Rightarrow \tan\tilde\chi = \frac{V_y^* - V_y(t_0) + g(t_f - t_0)}{V_x^* - V_x(t_0)}
\label{eq:three}
\end{equation}

This equation now only depends on $t_f - t_0$, which is called the Time to Go: $t_{go}$. We will use this equation to determine the value for $\tilde\chi$ given a value for 
$t_{go}$ \newline
We now consider again the linear tangent law, with the full constraint on final vertical position (i.e. the transversality condition for $P_y$ does not apply), the bilinear 
tangent law is now the linear tangent law, where the constants $c_2, c_3 $ and $c_4$ are recast to $k1,k_0$. We consider the tangenet law not at a specific time $t$ but at 

\begin{equation}
\tan\beta = k_1()
\end{equation}

\end{document}	